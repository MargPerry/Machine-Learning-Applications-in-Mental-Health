\documentclass[]{article}
\usepackage{lmodern}
\usepackage{amssymb,amsmath}
\usepackage{ifxetex,ifluatex}
\usepackage{fixltx2e} % provides \textsubscript
\ifnum 0\ifxetex 1\fi\ifluatex 1\fi=0 % if pdftex
  \usepackage[T1]{fontenc}
  \usepackage[utf8]{inputenc}
  \usepackage{eurosym}
\else % if luatex or xelatex
  \ifxetex
    \usepackage{mathspec}
  \else
    \usepackage{fontspec}
  \fi
  \defaultfontfeatures{Ligatures=TeX,Scale=MatchLowercase}
  \newcommand{\euro}{€}
\fi
% use upquote if available, for straight quotes in verbatim environments
\IfFileExists{upquote.sty}{\usepackage{upquote}}{}
% use microtype if available
\IfFileExists{microtype.sty}{%
\usepackage{microtype}
\UseMicrotypeSet[protrusion]{basicmath} % disable protrusion for tt fonts
}{}
\usepackage[margin=1in]{geometry}
\usepackage{hyperref}
\hypersetup{unicode=true,
            pdftitle={Main code},
            pdfauthor={Margaret Perry},
            pdfborder={0 0 0},
            breaklinks=true}
\urlstyle{same}  % don't use monospace font for urls
\usepackage{color}
\usepackage{fancyvrb}
\newcommand{\VerbBar}{|}
\newcommand{\VERB}{\Verb[commandchars=\\\{\}]}
\DefineVerbatimEnvironment{Highlighting}{Verbatim}{commandchars=\\\{\}}
% Add ',fontsize=\small' for more characters per line
\usepackage{framed}
\definecolor{shadecolor}{RGB}{248,248,248}
\newenvironment{Shaded}{\begin{snugshade}}{\end{snugshade}}
\newcommand{\KeywordTok}[1]{\textcolor[rgb]{0.13,0.29,0.53}{\textbf{#1}}}
\newcommand{\DataTypeTok}[1]{\textcolor[rgb]{0.13,0.29,0.53}{#1}}
\newcommand{\DecValTok}[1]{\textcolor[rgb]{0.00,0.00,0.81}{#1}}
\newcommand{\BaseNTok}[1]{\textcolor[rgb]{0.00,0.00,0.81}{#1}}
\newcommand{\FloatTok}[1]{\textcolor[rgb]{0.00,0.00,0.81}{#1}}
\newcommand{\ConstantTok}[1]{\textcolor[rgb]{0.00,0.00,0.00}{#1}}
\newcommand{\CharTok}[1]{\textcolor[rgb]{0.31,0.60,0.02}{#1}}
\newcommand{\SpecialCharTok}[1]{\textcolor[rgb]{0.00,0.00,0.00}{#1}}
\newcommand{\StringTok}[1]{\textcolor[rgb]{0.31,0.60,0.02}{#1}}
\newcommand{\VerbatimStringTok}[1]{\textcolor[rgb]{0.31,0.60,0.02}{#1}}
\newcommand{\SpecialStringTok}[1]{\textcolor[rgb]{0.31,0.60,0.02}{#1}}
\newcommand{\ImportTok}[1]{#1}
\newcommand{\CommentTok}[1]{\textcolor[rgb]{0.56,0.35,0.01}{\textit{#1}}}
\newcommand{\DocumentationTok}[1]{\textcolor[rgb]{0.56,0.35,0.01}{\textbf{\textit{#1}}}}
\newcommand{\AnnotationTok}[1]{\textcolor[rgb]{0.56,0.35,0.01}{\textbf{\textit{#1}}}}
\newcommand{\CommentVarTok}[1]{\textcolor[rgb]{0.56,0.35,0.01}{\textbf{\textit{#1}}}}
\newcommand{\OtherTok}[1]{\textcolor[rgb]{0.56,0.35,0.01}{#1}}
\newcommand{\FunctionTok}[1]{\textcolor[rgb]{0.00,0.00,0.00}{#1}}
\newcommand{\VariableTok}[1]{\textcolor[rgb]{0.00,0.00,0.00}{#1}}
\newcommand{\ControlFlowTok}[1]{\textcolor[rgb]{0.13,0.29,0.53}{\textbf{#1}}}
\newcommand{\OperatorTok}[1]{\textcolor[rgb]{0.81,0.36,0.00}{\textbf{#1}}}
\newcommand{\BuiltInTok}[1]{#1}
\newcommand{\ExtensionTok}[1]{#1}
\newcommand{\PreprocessorTok}[1]{\textcolor[rgb]{0.56,0.35,0.01}{\textit{#1}}}
\newcommand{\AttributeTok}[1]{\textcolor[rgb]{0.77,0.63,0.00}{#1}}
\newcommand{\RegionMarkerTok}[1]{#1}
\newcommand{\InformationTok}[1]{\textcolor[rgb]{0.56,0.35,0.01}{\textbf{\textit{#1}}}}
\newcommand{\WarningTok}[1]{\textcolor[rgb]{0.56,0.35,0.01}{\textbf{\textit{#1}}}}
\newcommand{\AlertTok}[1]{\textcolor[rgb]{0.94,0.16,0.16}{#1}}
\newcommand{\ErrorTok}[1]{\textcolor[rgb]{0.64,0.00,0.00}{\textbf{#1}}}
\newcommand{\NormalTok}[1]{#1}
\usepackage{graphicx,grffile}
\makeatletter
\def\maxwidth{\ifdim\Gin@nat@width>\linewidth\linewidth\else\Gin@nat@width\fi}
\def\maxheight{\ifdim\Gin@nat@height>\textheight\textheight\else\Gin@nat@height\fi}
\makeatother
% Scale images if necessary, so that they will not overflow the page
% margins by default, and it is still possible to overwrite the defaults
% using explicit options in \includegraphics[width, height, ...]{}
\setkeys{Gin}{width=\maxwidth,height=\maxheight,keepaspectratio}
\IfFileExists{parskip.sty}{%
\usepackage{parskip}
}{% else
\setlength{\parindent}{0pt}
\setlength{\parskip}{6pt plus 2pt minus 1pt}
}
\setlength{\emergencystretch}{3em}  % prevent overfull lines
\providecommand{\tightlist}{%
  \setlength{\itemsep}{0pt}\setlength{\parskip}{0pt}}
\setcounter{secnumdepth}{0}
% Redefines (sub)paragraphs to behave more like sections
\ifx\paragraph\undefined\else
\let\oldparagraph\paragraph
\renewcommand{\paragraph}[1]{\oldparagraph{#1}\mbox{}}
\fi
\ifx\subparagraph\undefined\else
\let\oldsubparagraph\subparagraph
\renewcommand{\subparagraph}[1]{\oldsubparagraph{#1}\mbox{}}
\fi

%%% Use protect on footnotes to avoid problems with footnotes in titles
\let\rmarkdownfootnote\footnote%
\def\footnote{\protect\rmarkdownfootnote}

%%% Change title format to be more compact
\usepackage{titling}

% Create subtitle command for use in maketitle
\newcommand{\subtitle}[1]{
  \posttitle{
    \begin{center}\large#1\end{center}
    }
}

\setlength{\droptitle}{-2em}

  \title{Main code}
    \pretitle{\vspace{\droptitle}\centering\huge}
  \posttitle{\par}
    \author{Margaret Perry}
    \preauthor{\centering\large\emph}
  \postauthor{\par}
      \predate{\centering\large\emph}
  \postdate{\par}
    \date{November 16, 2018}


\begin{document}
\maketitle

From the numerical summaries of these variables from the Suicide subset
of the data we can see that there are some categorical variables, but
most are numerical that depend on a larger range of numbers, and in this
case the numbers represent the ages of the people involved in the
studies. With our data we learn that most people said ``no'' to
seriously committing suicide both in the last 12 months and at all. But
from the other variables we can see that the people that have thought
about committing suicide had these thoughts in their 20s. There are many
NA values, so the summaries don't represent all of the observations from
the data.

We also realize that we need to convert some numeric values to binary
categorical values, for instance, there are two numbers, 1(no) and
5(yes), are used for indicating ICD alcohol dependence (lifetime). But
we believe it will be confusing to use it as numeric and we want to
convert it into a categorical variable. This is important to make sure
about our consistency in treating variables with two categories as
categorical, and treating variables with multiple categories as numeric.
There is a huge inconsistency in the values available for observations
used in different categories. For instance, it is hard to compare the
data from variables under category ``eating disorder'' with variables
from category ``family cohesion'' due to two datasets being actually
collected from two separate surveys. Even though these two datasets were
compiled together, it shows that the observations from one survey do not
necessarily include variables from another survey. Therefore, we believe
that it is essential for us to go back to our cross-walk original
dataset again to select variables that are all from the same survey,
instead of the collaborative survey that contains observations from
other surveys.

The same problem is also shown when we tried to use tree decision as our
main model. We were very excited to use tree decision model, however,
our model turned out to be a disaster. We only got one node and aren't
sure the cause. We believe that it might be due to missing values in ou
data. Our data worked well with KNN, mainly because KNN does not have
any assumptions. The only obstacle was that we were required to filter
out observations that had any missing values, however this still left us
with over 1,500 for the method. As for logistic regression, our data met
the assumption that the response variable is binary. We also tested on
large enough data (a sample of 1670). However, we are concerned about
potential multicollinearity, such as seriously contemplating suicide and
actually attempting suicide. It may suffice that we pick only one of
these variables.

We believe that we are having major issues with the variables we are
using right now. Many numeric variables containing only two values
should be converted into categorical variables. And we need to revisit
our initial category and variable selection by making sure all variables
are from the same survey, instead of the collaborative survey. Even
though it is painful to acknowledge this mistake we made earlier on
about variable selection, we have learned a lot about reading and
understanding complex collaborative survey datasets. And then we will
try tree decision or logistic model again.  Works Cited
\url{https://www.statisticssolutions.com/assumptions-of-logistic-regression/}

\begin{Shaded}
\begin{Highlighting}[]
\KeywordTok{library}\NormalTok{(haven)}
\KeywordTok{library}\NormalTok{(tidyr)}
\KeywordTok{library}\NormalTok{(dplyr)}
\end{Highlighting}
\end{Shaded}

\begin{verbatim}
## 
## Attaching package: 'dplyr'
\end{verbatim}

\begin{verbatim}
## The following objects are masked from 'package:stats':
## 
##     filter, lag
\end{verbatim}

\begin{verbatim}
## The following objects are masked from 'package:base':
## 
##     intersect, setdiff, setequal, union
\end{verbatim}

\begin{Shaded}
\begin{Highlighting}[]
\KeywordTok{library}\NormalTok{(mosaic)}
\end{Highlighting}
\end{Shaded}

\begin{verbatim}
## Loading required package: lattice
\end{verbatim}

\begin{verbatim}
## Loading required package: ggformula
\end{verbatim}

\begin{verbatim}
## Loading required package: ggplot2
\end{verbatim}

\begin{verbatim}
## Loading required package: ggstance
\end{verbatim}

\begin{verbatim}
## 
## Attaching package: 'ggstance'
\end{verbatim}

\begin{verbatim}
## The following objects are masked from 'package:ggplot2':
## 
##     geom_errorbarh, GeomErrorbarh
\end{verbatim}

\begin{verbatim}
## 
## New to ggformula?  Try the tutorials: 
##  learnr::run_tutorial("introduction", package = "ggformula")
##  learnr::run_tutorial("refining", package = "ggformula")
\end{verbatim}

\begin{verbatim}
## Loading required package: mosaicData
\end{verbatim}

\begin{verbatim}
## Loading required package: Matrix
\end{verbatim}

\begin{verbatim}
## 
## Attaching package: 'Matrix'
\end{verbatim}

\begin{verbatim}
## The following object is masked from 'package:tidyr':
## 
##     expand
\end{verbatim}

\begin{verbatim}
## 
## The 'mosaic' package masks several functions from core packages in order to add 
## additional features.  The original behavior of these functions should not be affected by this.
## 
## Note: If you use the Matrix package, be sure to load it BEFORE loading mosaic.
\end{verbatim}

\begin{verbatim}
## 
## Attaching package: 'mosaic'
\end{verbatim}

\begin{verbatim}
## The following object is masked from 'package:Matrix':
## 
##     mean
\end{verbatim}

\begin{verbatim}
## The following object is masked from 'package:ggplot2':
## 
##     stat
\end{verbatim}

\begin{verbatim}
## The following objects are masked from 'package:dplyr':
## 
##     count, do, tally
\end{verbatim}

\begin{verbatim}
## The following objects are masked from 'package:stats':
## 
##     binom.test, cor, cor.test, cov, fivenum, IQR, median,
##     prop.test, quantile, sd, t.test, var
\end{verbatim}

\begin{verbatim}
## The following objects are masked from 'package:base':
## 
##     max, mean, min, prod, range, sample, sum
\end{verbatim}

\begin{Shaded}
\begin{Highlighting}[]
\KeywordTok{library}\NormalTok{(tree)}
\KeywordTok{library}\NormalTok{(ISLR)}
\KeywordTok{library}\NormalTok{(ggplot2)}
\KeywordTok{library}\NormalTok{(class)}
\end{Highlighting}
\end{Shaded}

\subsection{Names (eating disorder)}\label{names-eating-disorder}

\begin{Shaded}
\begin{Highlighting}[]
\KeywordTok{library}\NormalTok{(readr)}
\KeywordTok{library}\NormalTok{(haven)}
\NormalTok{raw_data <-}\StringTok{ }\KeywordTok{read_sav}\NormalTok{(}\StringTok{"C:/Users/rocio/Downloads/20240-0001-Data.sav"}\NormalTok{)}
\NormalTok{eatdisord_}\DecValTok{1}\NormalTok{ <-}\StringTok{ }\KeywordTok{read_csv}\NormalTok{(}\StringTok{"Variblenamefiles/eatdisord_1.csv"}\NormalTok{)}
\end{Highlighting}
\end{Shaded}

\begin{verbatim}
## Warning: Missing column names filled in: 'X3' [3], 'X4' [4], 'X5' [5],
## 'X6' [6], 'X7' [7], 'X8' [8], 'X9' [9], 'X10' [10], 'X11' [11], 'X12' [12],
## 'X13' [13], 'X14' [14], 'X15' [15], 'X16' [16], 'X17' [17], 'X18' [18],
## 'X19' [19], 'X20' [20], 'X21' [21], 'X22' [22], 'X23' [23]
\end{verbatim}

\begin{verbatim}
## Parsed with column specification:
## cols(
##   .default = col_character()
## )
\end{verbatim}

\begin{verbatim}
## See spec(...) for full column specifications.
\end{verbatim}

\begin{Shaded}
\begin{Highlighting}[]
\NormalTok{sub_use_}\DecValTok{1}\NormalTok{ <-}\StringTok{ }\KeywordTok{read_csv}\NormalTok{(}\StringTok{"Variblenamefiles/sub_use_1.csv"}\NormalTok{)}
\end{Highlighting}
\end{Shaded}

\begin{verbatim}
## Warning: Missing column names filled in: 'X3' [3], 'X4' [4], 'X5' [5],
## 'X6' [6], 'X7' [7], 'X8' [8], 'X9' [9], 'X10' [10], 'X11' [11], 'X12' [12],
## 'X13' [13], 'X14' [14], 'X15' [15], 'X16' [16], 'X17' [17], 'X18' [18],
## 'X19' [19], 'X20' [20], 'X21' [21], 'X22' [22], 'X23' [23]
\end{verbatim}

\begin{verbatim}
## Parsed with column specification:
## cols(
##   .default = col_character()
## )
## See spec(...) for full column specifications.
\end{verbatim}

\begin{Shaded}
\begin{Highlighting}[]
\NormalTok{socnet<-}\StringTok{ }\KeywordTok{read_csv}\NormalTok{(}\StringTok{"Variblenamefiles/socialnetworks.csv"}\NormalTok{)}
\end{Highlighting}
\end{Shaded}

\begin{verbatim}
## Warning: Missing column names filled in: 'X3' [3], 'X4' [4], 'X5' [5],
## 'X6' [6], 'X7' [7], 'X8' [8], 'X9' [9], 'X10' [10], 'X11' [11], 'X12' [12],
## 'X13' [13], 'X14' [14], 'X15' [15], 'X16' [16], 'X17' [17], 'X18' [18],
## 'X19' [19], 'X20' [20], 'X21' [21], 'X22' [22], 'X23' [23], 'X24' [24],
## 'X25' [25], 'X26' [26]
\end{verbatim}

\begin{verbatim}
## Parsed with column specification:
## cols(
##   .default = col_character()
## )
## See spec(...) for full column specifications.
\end{verbatim}

\begin{Shaded}
\begin{Highlighting}[]
\NormalTok{tobacco_}\DecValTok{1}\NormalTok{<-}\StringTok{ }\KeywordTok{read_csv}\NormalTok{(}\StringTok{"Variblenamefiles/Tobacco_1.csv"}\NormalTok{)}
\end{Highlighting}
\end{Shaded}

\begin{verbatim}
## Warning: Missing column names filled in: 'X3' [3], 'X4' [4], 'X5' [5],
## 'X6' [6], 'X7' [7], 'X8' [8], 'X9' [9], 'X10' [10], 'X11' [11], 'X12' [12],
## 'X13' [13], 'X14' [14], 'X15' [15], 'X16' [16], 'X17' [17], 'X18' [18],
## 'X19' [19], 'X20' [20], 'X21' [21], 'X22' [22], 'X23' [23]
\end{verbatim}

\begin{verbatim}
## Parsed with column specification:
## cols(
##   .default = col_character()
## )
## See spec(...) for full column specifications.
\end{verbatim}

\begin{Shaded}
\begin{Highlighting}[]
\NormalTok{alcohol<-}\StringTok{ }\KeywordTok{read_csv}\NormalTok{(}\StringTok{"Variblenamefiles/alcohol.csv"}\NormalTok{)}
\end{Highlighting}
\end{Shaded}

\begin{verbatim}
## Warning: Missing column names filled in: 'X3' [3], 'X4' [4], 'X5' [5],
## 'X6' [6], 'X7' [7], 'X8' [8], 'X9' [9], 'X10' [10], 'X11' [11], 'X12' [12],
## 'X13' [13], 'X14' [14], 'X15' [15], 'X16' [16], 'X17' [17], 'X18' [18],
## 'X19' [19], 'X20' [20], 'X21' [21], 'X22' [22], 'X23' [23], 'X24' [24],
## 'X25' [25], 'X26' [26]
\end{verbatim}

\begin{verbatim}
## Parsed with column specification:
## cols(
##   .default = col_character()
## )
## See spec(...) for full column specifications.
\end{verbatim}

\begin{Shaded}
\begin{Highlighting}[]
\NormalTok{social_ph<-}\StringTok{ }\KeywordTok{read_csv}\NormalTok{(}\StringTok{"Variblenamefiles/Soicalph_1.csv"}\NormalTok{)}
\end{Highlighting}
\end{Shaded}

\begin{verbatim}
## Warning: Missing column names filled in: 'X3' [3], 'X4' [4], 'X5' [5],
## 'X6' [6], 'X7' [7], 'X8' [8], 'X9' [9], 'X10' [10], 'X11' [11], 'X12' [12],
## 'X13' [13], 'X14' [14], 'X15' [15], 'X16' [16], 'X17' [17], 'X18' [18],
## 'X19' [19], 'X20' [20], 'X21' [21], 'X22' [22], 'X23' [23]
\end{verbatim}

\begin{verbatim}
## Parsed with column specification:
## cols(
##   .default = col_character()
## )
## See spec(...) for full column specifications.
\end{verbatim}

\begin{Shaded}
\begin{Highlighting}[]
\NormalTok{persona<-}\StringTok{ }\KeywordTok{read_csv}\NormalTok{(}\StringTok{"Variblenamefiles/personality_1.csv"}\NormalTok{)}
\end{Highlighting}
\end{Shaded}

\begin{verbatim}
## Warning: Missing column names filled in: 'X3' [3], 'X4' [4], 'X5' [5],
## 'X6' [6], 'X7' [7], 'X8' [8], 'X9' [9], 'X10' [10], 'X11' [11], 'X12' [12],
## 'X13' [13], 'X14' [14], 'X15' [15], 'X16' [16], 'X17' [17], 'X18' [18],
## 'X19' [19], 'X20' [20], 'X21' [21], 'X22' [22], 'X23' [23], 'X24' [24],
## 'X25' [25], 'X26' [26]
\end{verbatim}

\begin{verbatim}
## Parsed with column specification:
## cols(
##   .default = col_character()
## )
## See spec(...) for full column specifications.
\end{verbatim}

\begin{Shaded}
\begin{Highlighting}[]
\NormalTok{marriage<-}\StringTok{ }\KeywordTok{read_csv}\NormalTok{(}\StringTok{"Variblenamefiles/marriage.csv"}\NormalTok{)}
\end{Highlighting}
\end{Shaded}

\begin{verbatim}
## Warning: Missing column names filled in: 'X3' [3], 'X4' [4], 'X5' [5],
## 'X6' [6], 'X7' [7], 'X8' [8], 'X9' [9], 'X10' [10], 'X11' [11], 'X12' [12],
## 'X13' [13], 'X14' [14], 'X15' [15], 'X16' [16], 'X17' [17], 'X18' [18],
## 'X19' [19], 'X20' [20], 'X21' [21], 'X22' [22], 'X23' [23], 'X24' [24],
## 'X25' [25], 'X26' [26]
\end{verbatim}

\begin{verbatim}
## Parsed with column specification:
## cols(
##   .default = col_character()
## )
## See spec(...) for full column specifications.
\end{verbatim}

\begin{Shaded}
\begin{Highlighting}[]
\NormalTok{ner_at<-}\StringTok{ }\KeywordTok{read_csv}\NormalTok{(}\StringTok{"Variblenamefiles/nervous_attack.csv"}\NormalTok{)}
\end{Highlighting}
\end{Shaded}

\begin{verbatim}
## Warning: Missing column names filled in: 'X3' [3], 'X4' [4], 'X5' [5],
## 'X6' [6], 'X7' [7], 'X8' [8], 'X9' [9], 'X10' [10], 'X11' [11], 'X12' [12],
## 'X13' [13], 'X14' [14], 'X15' [15], 'X16' [16], 'X17' [17], 'X18' [18],
## 'X19' [19], 'X20' [20], 'X21' [21], 'X22' [22], 'X23' [23], 'X24' [24],
## 'X25' [25], 'X26' [26]
\end{verbatim}

\begin{verbatim}
## Parsed with column specification:
## cols(
##   .default = col_character()
## )
## See spec(...) for full column specifications.
\end{verbatim}

\begin{Shaded}
\begin{Highlighting}[]
\NormalTok{discrim<-}\StringTok{ }\KeywordTok{read_csv}\NormalTok{(}\StringTok{"Variblenamefiles/discrimination.csv"}\NormalTok{)}
\end{Highlighting}
\end{Shaded}

\begin{verbatim}
## Warning: Missing column names filled in: 'X3' [3], 'X4' [4], 'X5' [5],
## 'X6' [6], 'X7' [7], 'X8' [8], 'X9' [9], 'X10' [10], 'X11' [11], 'X12' [12],
## 'X13' [13], 'X14' [14], 'X15' [15], 'X16' [16], 'X17' [17], 'X18' [18],
## 'X19' [19], 'X20' [20], 'X21' [21], 'X22' [22], 'X23' [23], 'X24' [24],
## 'X25' [25], 'X26' [26]
\end{verbatim}

\begin{verbatim}
## Parsed with column specification:
## cols(
##   .default = col_character()
## )
## See spec(...) for full column specifications.
\end{verbatim}

\begin{Shaded}
\begin{Highlighting}[]
\NormalTok{demograph<-}\StringTok{ }\KeywordTok{read_csv}\NormalTok{(}\StringTok{"Variblenamefiles/demographics.csv"}\NormalTok{)}
\end{Highlighting}
\end{Shaded}

\begin{verbatim}
## Warning: Missing column names filled in: 'X3' [3], 'X4' [4], 'X5' [5],
## 'X6' [6], 'X7' [7], 'X8' [8], 'X9' [9], 'X10' [10], 'X11' [11], 'X12' [12],
## 'X13' [13], 'X14' [14], 'X15' [15], 'X16' [16], 'X17' [17], 'X18' [18],
## 'X19' [19], 'X20' [20], 'X21' [21], 'X22' [22], 'X23' [23], 'X24' [24],
## 'X25' [25], 'X26' [26]
\end{verbatim}

\begin{verbatim}
## Parsed with column specification:
## cols(
##   .default = col_character()
## )
## See spec(...) for full column specifications.
\end{verbatim}

\begin{Shaded}
\begin{Highlighting}[]
\NormalTok{family<-}\StringTok{ }\KeywordTok{read_csv}\NormalTok{(}\StringTok{"Variblenamefiles/family.csv"}\NormalTok{)}
\end{Highlighting}
\end{Shaded}

\begin{verbatim}
## Warning: Missing column names filled in: 'X3' [3], 'X4' [4], 'X5' [5],
## 'X6' [6], 'X7' [7], 'X8' [8], 'X9' [9], 'X10' [10], 'X11' [11], 'X12' [12],
## 'X13' [13], 'X14' [14], 'X15' [15], 'X16' [16], 'X17' [17], 'X18' [18],
## 'X19' [19], 'X20' [20], 'X21' [21], 'X22' [22], 'X23' [23], 'X24' [24],
## 'X25' [25], 'X26' [26]
\end{verbatim}

\begin{verbatim}
## Parsed with column specification:
## cols(
##   .default = col_character()
## )
## See spec(...) for full column specifications.
\end{verbatim}

\begin{Shaded}
\begin{Highlighting}[]
\NormalTok{family_n_friends<-}\StringTok{ }\KeywordTok{read_csv}\NormalTok{(}\StringTok{"Variblenamefiles/family_n_friends.csv"}\NormalTok{)}
\end{Highlighting}
\end{Shaded}

\begin{verbatim}
## Warning: Missing column names filled in: 'X3' [3], 'X4' [4], 'X5' [5],
## 'X6' [6], 'X7' [7], 'X8' [8], 'X9' [9], 'X10' [10], 'X11' [11], 'X12' [12],
## 'X13' [13], 'X14' [14], 'X15' [15], 'X16' [16], 'X17' [17], 'X18' [18],
## 'X19' [19], 'X20' [20], 'X21' [21], 'X22' [22], 'X23' [23], 'X24' [24],
## 'X25' [25], 'X26' [26]
\end{verbatim}

\begin{verbatim}
## Parsed with column specification:
## cols(
##   .default = col_character()
## )
## See spec(...) for full column specifications.
\end{verbatim}

\begin{Shaded}
\begin{Highlighting}[]
\NormalTok{depress<-}\StringTok{ }\KeywordTok{read_csv}\NormalTok{(}\StringTok{"Variblenamefiles/depression.csv"}\NormalTok{)}
\end{Highlighting}
\end{Shaded}

\begin{verbatim}
## Warning: Missing column names filled in: 'X3' [3], 'X4' [4], 'X5' [5],
## 'X6' [6], 'X7' [7], 'X8' [8], 'X9' [9], 'X10' [10], 'X11' [11], 'X12' [12],
## 'X13' [13], 'X14' [14], 'X15' [15], 'X16' [16], 'X17' [17], 'X18' [18],
## 'X19' [19], 'X20' [20], 'X21' [21], 'X22' [22], 'X23' [23], 'X24' [24],
## 'X25' [25], 'X26' [26]
\end{verbatim}

\begin{verbatim}
## Parsed with column specification:
## cols(
##   .default = col_character()
## )
## See spec(...) for full column specifications.
\end{verbatim}

\begin{Shaded}
\begin{Highlighting}[]
\NormalTok{suicide<-}\StringTok{ }\KeywordTok{read_csv}\NormalTok{(}\StringTok{"Variblenamefiles/Suicide.csv"}\NormalTok{)}
\end{Highlighting}
\end{Shaded}

\begin{verbatim}
## Warning: Missing column names filled in: 'X3' [3], 'X4' [4], 'X5' [5],
## 'X6' [6], 'X7' [7], 'X8' [8], 'X9' [9], 'X10' [10], 'X11' [11], 'X12' [12],
## 'X13' [13], 'X14' [14], 'X15' [15], 'X16' [16], 'X17' [17], 'X18' [18],
## 'X19' [19], 'X20' [20], 'X21' [21], 'X22' [22], 'X23' [23]
\end{verbatim}

\begin{verbatim}
## Parsed with column specification:
## cols(
##   .default = col_character()
## )
## See spec(...) for full column specifications.
\end{verbatim}

\begin{Shaded}
\begin{Highlighting}[]
\StringTok{"Uncomment when uploading raw data however the raw dataset is too large for Git"}
\end{Highlighting}
\end{Shaded}

\begin{Shaded}
\begin{Highlighting}[]
\CommentTok{#Read in Var names}

\CommentTok{#Main}
\NormalTok{eatdisord_}\DecValTok{1}\NormalTok{<-}\StringTok{ }\KeywordTok{select}\NormalTok{(eatdisord_}\DecValTok{1}\NormalTok{, CPES, Eating_Disorder)}
\NormalTok{eatdisord_}\DecValTok{1}\NormalTok{=}\StringTok{ }\KeywordTok{drop_na}\NormalTok{(eatdisord_}\DecValTok{1}\NormalTok{)}
\KeywordTok{View}\NormalTok{(eatdisord_}\DecValTok{1}\NormalTok{)}

\CommentTok{#CPES}
\NormalTok{CPES_ed=}\StringTok{ }\KeywordTok{select}\NormalTok{(eatdisord_}\DecValTok{1}\NormalTok{, CPES)}
\KeywordTok{View}\NormalTok{(CPES_ed)}

\CommentTok{#Full_explaination}
\NormalTok{full_name=}\StringTok{ }\KeywordTok{select}\NormalTok{(eatdisord_}\DecValTok{1}\NormalTok{, Eating_Disorder)}
\KeywordTok{View}\NormalTok{(full_name)}

\CommentTok{#Make it a list}
\NormalTok{namelist =}\StringTok{ }\KeywordTok{as.list}\NormalTok{(full_name)}
\StringTok{'print(namelist)'}
\StringTok{' We did not need this and can join on just the CPES code and maintain the information of the varible'}
\end{Highlighting}
\end{Shaded}

\subsection{Making Subset(eating
disorder)}\label{making-subseteating-disorder}

\begin{Shaded}
\begin{Highlighting}[]
\NormalTok{data_ed=}\StringTok{ }\KeywordTok{select}\NormalTok{(raw_data, CPES_ed}\OperatorTok{$}\NormalTok{CPES[}\DecValTok{1}\OperatorTok{:}\DecValTok{64}\NormalTok{])}
\KeywordTok{View}\NormalTok{(data_ed)}
\end{Highlighting}
\end{Shaded}

\section{Substance varibles}\label{substance-varibles}

\begin{Shaded}
\begin{Highlighting}[]
\CommentTok{#Main}
\NormalTok{sub_use_}\DecValTok{1}\NormalTok{<-}\StringTok{ }\KeywordTok{select}\NormalTok{(sub_use_}\DecValTok{1}\NormalTok{, CPES)}
\NormalTok{sub_use_}\DecValTok{1}\NormalTok{=}\StringTok{ }\KeywordTok{drop_na}\NormalTok{(sub_use_}\DecValTok{1}\NormalTok{)}
\CommentTok{#View(sub_use_1)}

\CommentTok{#Make it a list}
\NormalTok{namelist_su =}\StringTok{ }\KeywordTok{as.list}\NormalTok{(sub_use_}\DecValTok{1}\NormalTok{)}
\NormalTok{data_su=}\StringTok{ }\KeywordTok{select}\NormalTok{(raw_data, sub_use_}\DecValTok{1}\OperatorTok{$}\NormalTok{CPES[}\DecValTok{1}\OperatorTok{:}\DecValTok{70}\NormalTok{])}
\end{Highlighting}
\end{Shaded}

\section{Tobacco}\label{tobacco}

\begin{Shaded}
\begin{Highlighting}[]
\CommentTok{#Main}
\NormalTok{tobacco_}\DecValTok{1}\NormalTok{<-}\StringTok{ }\KeywordTok{select}\NormalTok{(tobacco_}\DecValTok{1}\NormalTok{, CPES)}
\NormalTok{tobacco_}\DecValTok{1}\NormalTok{=}\StringTok{ }\KeywordTok{drop_na}\NormalTok{(tobacco_}\DecValTok{1}\NormalTok{)}
\CommentTok{#Make it a list}
\NormalTok{namelist_to =}\StringTok{ }\KeywordTok{as.list}\NormalTok{(tobacco_}\DecValTok{1}\NormalTok{)}
\NormalTok{data_to=}\StringTok{ }\KeywordTok{select}\NormalTok{(raw_data, tobacco_}\DecValTok{1}\OperatorTok{$}\NormalTok{CPES[}\DecValTok{1}\OperatorTok{:}\DecValTok{65}\NormalTok{])}
\end{Highlighting}
\end{Shaded}

\section{Social Phobia}\label{social-phobia}

\begin{Shaded}
\begin{Highlighting}[]
\CommentTok{#Main}
\NormalTok{social_ph<-}\StringTok{ }\KeywordTok{select}\NormalTok{(social_ph, CPES)}
\NormalTok{social_ph=}\StringTok{ }\KeywordTok{drop_na}\NormalTok{(social_ph)}
\CommentTok{#Make it a list}
\NormalTok{namelist_sph =}\StringTok{ }\KeywordTok{as.list}\NormalTok{(social_ph)}
\NormalTok{data_sph=}\StringTok{ }\KeywordTok{select}\NormalTok{(raw_data, social_ph}\OperatorTok{$}\NormalTok{CPES[}\DecValTok{1}\OperatorTok{:}\DecValTok{87}\NormalTok{])}
\end{Highlighting}
\end{Shaded}

\section{Personality}\label{personality}

\begin{Shaded}
\begin{Highlighting}[]
\CommentTok{#Main}
\NormalTok{persona<-}\StringTok{ }\KeywordTok{select}\NormalTok{(persona, CPES)}
\NormalTok{persona=}\StringTok{ }\KeywordTok{drop_na}\NormalTok{(persona)}
\CommentTok{#Make it a list}
\NormalTok{namelist_per =}\StringTok{ }\KeywordTok{as.list}\NormalTok{(persona)}
\NormalTok{data_per=}\StringTok{ }\KeywordTok{select}\NormalTok{(raw_data, namelist_per}\OperatorTok{$}\NormalTok{CPES[}\DecValTok{1}\OperatorTok{:}\DecValTok{44}\NormalTok{])}
\end{Highlighting}
\end{Shaded}

\section{Marriage}\label{marriage}

\begin{Shaded}
\begin{Highlighting}[]
\CommentTok{#Main}
\NormalTok{marriage<-}\StringTok{ }\KeywordTok{select}\NormalTok{(marriage, CPES)}
\NormalTok{marriage=}\StringTok{ }\KeywordTok{drop_na}\NormalTok{(marriage)}
\CommentTok{#Make it a list}
\NormalTok{namelist_mar =}\StringTok{ }\KeywordTok{as.list}\NormalTok{(marriage)}
\NormalTok{data_mar=}\StringTok{ }\KeywordTok{select}\NormalTok{(raw_data, namelist_mar}\OperatorTok{$}\NormalTok{CPES[}\DecValTok{1}\OperatorTok{:}\DecValTok{29}\NormalTok{])}
\end{Highlighting}
\end{Shaded}

\section{Social Networks}\label{social-networks}

\begin{Shaded}
\begin{Highlighting}[]
\CommentTok{#Main}
\NormalTok{socnet<-}\StringTok{ }\KeywordTok{select}\NormalTok{(socnet, CPES)}
\NormalTok{socnet=}\StringTok{ }\KeywordTok{drop_na}\NormalTok{(socnet)}
\CommentTok{#Make it a list}
\NormalTok{namelist_sn =}\StringTok{ }\KeywordTok{as.list}\NormalTok{(socnet)}
\NormalTok{data_sn=}\StringTok{ }\KeywordTok{select}\NormalTok{(raw_data, namelist_sn}\OperatorTok{$}\NormalTok{CPES[}\DecValTok{1}\OperatorTok{:}\DecValTok{15}\NormalTok{])}
\end{Highlighting}
\end{Shaded}

\section{Nervous Attack}\label{nervous-attack}

\begin{Shaded}
\begin{Highlighting}[]
\CommentTok{#Main}
\NormalTok{ner_at<-}\StringTok{ }\KeywordTok{select}\NormalTok{(ner_at, CPES)}
\NormalTok{ner_at=}\StringTok{ }\KeywordTok{drop_na}\NormalTok{(ner_at)}
\CommentTok{#Make it a list}
\NormalTok{namelist_nerv =}\StringTok{ }\KeywordTok{as.list}\NormalTok{(ner_at)}
\NormalTok{data_nerv=}\StringTok{ }\KeywordTok{select}\NormalTok{(raw_data, namelist_nerv}\OperatorTok{$}\NormalTok{CPES[}\DecValTok{1}\OperatorTok{:}\DecValTok{23}\NormalTok{])}
\end{Highlighting}
\end{Shaded}

\section{Discrimination}\label{discrimination}

\begin{Shaded}
\begin{Highlighting}[]
\CommentTok{#Main}
\NormalTok{discrim<-}\StringTok{ }\KeywordTok{select}\NormalTok{(discrim, CPES)}
\NormalTok{discrim=}\StringTok{ }\KeywordTok{drop_na}\NormalTok{(discrim)}
\CommentTok{#Make it a list}
\NormalTok{namelist_disc =}\StringTok{ }\KeywordTok{as.list}\NormalTok{(discrim)}
\NormalTok{data_disc=}\StringTok{ }\KeywordTok{select}\NormalTok{(raw_data, namelist_disc}\OperatorTok{$}\NormalTok{CPES[}\DecValTok{1}\OperatorTok{:}\DecValTok{13}\NormalTok{])}
\end{Highlighting}
\end{Shaded}

\section{Demographics}\label{demographics}

\begin{Shaded}
\begin{Highlighting}[]
\CommentTok{#Main}
\NormalTok{demograph<-}\StringTok{ }\KeywordTok{select}\NormalTok{(demograph, CPES)}
\NormalTok{demograph=}\StringTok{ }\KeywordTok{drop_na}\NormalTok{(demograph)}
\CommentTok{#Make it a list}
\NormalTok{namelist_dem=}\StringTok{ }\KeywordTok{as.list}\NormalTok{(demograph)}
\NormalTok{data_dem=}\StringTok{ }\KeywordTok{select}\NormalTok{(raw_data, namelist_dem}\OperatorTok{$}\NormalTok{CPES[}\DecValTok{1}\OperatorTok{:}\DecValTok{14}\NormalTok{])}
\end{Highlighting}
\end{Shaded}

\section{Family Cohesion}\label{family-cohesion}

\begin{Shaded}
\begin{Highlighting}[]
\CommentTok{#Main}
\NormalTok{family<-}\StringTok{ }\KeywordTok{select}\NormalTok{(family, CPES)}
\NormalTok{family=}\StringTok{ }\KeywordTok{drop_na}\NormalTok{(family)}
\CommentTok{#Make it a list}
\NormalTok{namelist_fam =}\StringTok{ }\KeywordTok{as.list}\NormalTok{(family)}
\NormalTok{data_fam=}\StringTok{ }\KeywordTok{select}\NormalTok{(raw_data, namelist_fam}\OperatorTok{$}\NormalTok{CPES[}\DecValTok{1}\OperatorTok{:}\DecValTok{15}\NormalTok{])}
\end{Highlighting}
\end{Shaded}

\section{Family and Friends}\label{family-and-friends}

\begin{Shaded}
\begin{Highlighting}[]
\CommentTok{#Main}
\NormalTok{family_n_friends<-}\StringTok{ }\KeywordTok{select}\NormalTok{(family_n_friends, CPES)}
\NormalTok{family_n_friends=}\StringTok{ }\KeywordTok{drop_na}\NormalTok{(family_n_friends)}
\CommentTok{#Make it a list}
\NormalTok{namelist_ff =}\StringTok{ }\KeywordTok{as.list}\NormalTok{(family_n_friends)}
\NormalTok{data_ff=}\StringTok{ }\KeywordTok{select}\NormalTok{(raw_data, namelist_ff}\OperatorTok{$}\NormalTok{CPES[}\DecValTok{1}\OperatorTok{:}\DecValTok{39}\NormalTok{])}
\end{Highlighting}
\end{Shaded}

\section{Depression Espoide}\label{depression-espoide}

\begin{Shaded}
\begin{Highlighting}[]
\CommentTok{#Main}
\NormalTok{depress<-}\StringTok{ }\KeywordTok{select}\NormalTok{(depress, CPES)}
\NormalTok{depress=}\StringTok{ }\KeywordTok{drop_na}\NormalTok{(depress)}
\CommentTok{#Make it a list}
\NormalTok{namelist_dep =}\StringTok{ }\KeywordTok{as.list}\NormalTok{(depress)}
\NormalTok{data_dep=}\StringTok{ }\KeywordTok{select}\NormalTok{(raw_data, namelist_dep}\OperatorTok{$}\NormalTok{CPES[}\DecValTok{1}\OperatorTok{:}\DecValTok{5}\NormalTok{])}
\end{Highlighting}
\end{Shaded}

\section{Alcohol}\label{alcohol}

\begin{Shaded}
\begin{Highlighting}[]
\CommentTok{#Main}
\NormalTok{alcohol<-}\StringTok{ }\KeywordTok{select}\NormalTok{(alcohol, CPES)}
\NormalTok{alcohol=}\StringTok{ }\KeywordTok{drop_na}\NormalTok{(alcohol)}
\CommentTok{#Make it a list}
\NormalTok{namelist_alco =}\StringTok{ }\KeywordTok{as.list}\NormalTok{(alcohol)}
\NormalTok{data_alco=}\StringTok{ }\KeywordTok{select}\NormalTok{(raw_data, namelist_alco}\OperatorTok{$}\NormalTok{CPES[}\DecValTok{1}\OperatorTok{:}\DecValTok{5}\NormalTok{])}
\end{Highlighting}
\end{Shaded}

\section{Suicide}\label{suicide}

\begin{Shaded}
\begin{Highlighting}[]
\CommentTok{#Main}
\NormalTok{suicide<-}\StringTok{ }\KeywordTok{select}\NormalTok{(suicide, CPES)}
\NormalTok{suicide=}\StringTok{ }\KeywordTok{drop_na}\NormalTok{(suicide)}
\CommentTok{#Make it a list}
\NormalTok{namelist_sui =}\StringTok{ }\KeywordTok{as.list}\NormalTok{(suicide)}
\NormalTok{data_sui=}\StringTok{ }\KeywordTok{select}\NormalTok{(raw_data, namelist_sui}\OperatorTok{$}\NormalTok{CPES[}\DecValTok{1}\OperatorTok{:}\DecValTok{12}\NormalTok{])}
\end{Highlighting}
\end{Shaded}

\section{Tree and Logistic Models}\label{tree-and-logistic-models}

\begin{Shaded}
\begin{Highlighting}[]
\CommentTok{#tree model using alcohol as a predictor}
\NormalTok{alco =}\StringTok{ }\KeywordTok{cbind}\NormalTok{(data_dep, data_alco)}
\NormalTok{alco}
\NormalTok{alco =}\StringTok{ }\NormalTok{alco}\OperatorTok
\StringTok{  }\KeywordTok{mutate}\NormalTok{(}\DataTypeTok{Depressed =} \KeywordTok{as.factor}\NormalTok{(}\KeywordTok{ifelse}\NormalTok{(V07876 }\OperatorTok{==}\StringTok{ }\DecValTok{1}\NormalTok{, }\DecValTok{1}\NormalTok{, }\DecValTok{0}\NormalTok{)))}
\KeywordTok{set.seed}\NormalTok{(}\DecValTok{1}\NormalTok{)}
\NormalTok{train =}\StringTok{ }\NormalTok{alco }\OperatorTok
\StringTok{  }\KeywordTok{sample_n}\NormalTok{(}\DecValTok{10000}\NormalTok{)}
\NormalTok{test =}\StringTok{ }\NormalTok{alco }\OperatorTok
\StringTok{  }\KeywordTok{setdiff}\NormalTok{(train)}
\end{Highlighting}
\end{Shaded}

\begin{verbatim}
## Warning: Column `V07657` has different attributes on LHS and RHS of join
\end{verbatim}

\begin{verbatim}
## Warning: Column `V07655` has different attributes on LHS and RHS of join
\end{verbatim}

\begin{verbatim}
## Warning: Column `V07876` has different attributes on LHS and RHS of join
\end{verbatim}

\begin{verbatim}
## Warning: Column `V08766` has different attributes on LHS and RHS of join
\end{verbatim}

\begin{verbatim}
## Warning: Column `V08768` has different attributes on LHS and RHS of join
\end{verbatim}

\begin{verbatim}
## Warning: Column `V08312` has different attributes on LHS and RHS of join
\end{verbatim}

\begin{verbatim}
## Warning: Column `V08311` has different attributes on LHS and RHS of join
\end{verbatim}

\begin{verbatim}
## Warning: Column `V08515` has different attributes on LHS and RHS of join
\end{verbatim}

\begin{verbatim}
## Warning: Column `V07342` has different attributes on LHS and RHS of join
\end{verbatim}

\begin{verbatim}
## Warning: Column `V07345` has different attributes on LHS and RHS of join
\end{verbatim}

\begin{Shaded}
\begin{Highlighting}[]
\NormalTok{tree_alco=}\KeywordTok{tree}\NormalTok{(Depressed}\OperatorTok{~}\NormalTok{., train)}
\NormalTok{tree_alco}
\KeywordTok{summary}\NormalTok{(tree_alco)}
\NormalTok{tree_alco}
\CommentTok{#plot(tree_alco)}
\CommentTok{#text(tree_alco)}

\CommentTok{#logistic regression}
\NormalTok{glm_fit =}\StringTok{ }\KeywordTok{glm}\NormalTok{(Depressed}\OperatorTok{~}\NormalTok{. }\OperatorTok{-}\StringTok{ }\NormalTok{V07876,}
\DataTypeTok{data =}\NormalTok{ train,}
\DataTypeTok{family =}\NormalTok{ binomial)}
\KeywordTok{summary}\NormalTok{(glm_fit)}
\KeywordTok{coefficients}\NormalTok{(glm_fit)}

\CommentTok{#a combination of depression, alcohol use, suicidality, and nervousness}
\CommentTok{#V01993: Seriously thought about committing suicide}
\CommentTok{#V02044: Ever attempted suicide}
\CommentTok{#V08312: ICD Alcohol Dependence (30 day)}
\NormalTok{combo1 <-}\StringTok{ }\KeywordTok{cbind}\NormalTok{(data_dep, data_alco, data_sui, data_nerv)}
\NormalTok{combo1 <-}\StringTok{ }\NormalTok{combo1}\OperatorTok
\StringTok{    }\KeywordTok{mutate}\NormalTok{(}\DataTypeTok{Depressed =} \KeywordTok{as.factor}\NormalTok{(}\KeywordTok{ifelse}\NormalTok{(V07876 }\OperatorTok{==}\StringTok{ }\DecValTok{1}\NormalTok{, }\DecValTok{1}\NormalTok{, }\DecValTok{0}\NormalTok{)))}


\CommentTok{#LOGISTIC REGRESSION}
\NormalTok{glm_fit =}\StringTok{ }\KeywordTok{glm}\NormalTok{(Depressed}\OperatorTok{~}\StringTok{ }\NormalTok{V01993 }\OperatorTok{+}\StringTok{ }\NormalTok{V02044 }\OperatorTok{+}\StringTok{ }\NormalTok{V08312,}
  \DataTypeTok{data =}\NormalTok{ combo1,}
  \DataTypeTok{family =}\NormalTok{ binomial)}
\KeywordTok{summary}\NormalTok{(glm_fit)}
\KeywordTok{coefficients}\NormalTok{(glm_fit)}
\end{Highlighting}
\end{Shaded}

\section{KNN using suicide, alcohol abuse, and
personality}\label{knn-using-suicide-alcohol-abuse-and-personality}

\begin{Shaded}
\begin{Highlighting}[]
\NormalTok{combo2 <-}\StringTok{ }\KeywordTok{cbind}\NormalTok{(data_alco, data_per, data_sui, data_dep)}\OperatorTok
\StringTok{  }\KeywordTok{select}\NormalTok{(V07876, V03234,V01993, V02044, V08312, V03256)}
\NormalTok{combo2 =}\StringTok{  }\KeywordTok{na.omit}\NormalTok{(combo2)}


\CommentTok{#V03234: Often feel empty inside}
\CommentTok{#V01993: Seriously thought about committing suicide}
\CommentTok{#V02044: Ever attempted suicide}
\CommentTok{#V08312: ICD Alcohol Dependence (30 day)}
\CommentTok{#V03256: Feel uncomfortabe/helpless when alone}


\NormalTok{train_obvs =}\StringTok{ }\NormalTok{combo2 }\OperatorTok
\StringTok{  }\KeywordTok{slice}\NormalTok{(}\DecValTok{1}\OperatorTok{:}\DecValTok{800}\NormalTok{) }\OperatorTok
\StringTok{  }\KeywordTok{select}\NormalTok{(V03234,V01993, V02044, V08312, V03256)}


\NormalTok{test_obvs =}\StringTok{ }\NormalTok{combo2 }\OperatorTok
\StringTok{  }\KeywordTok{slice}\NormalTok{(}\DecValTok{801}\OperatorTok{:}\KeywordTok{n}\NormalTok{())}\OperatorTok
\StringTok{  }\KeywordTok{select}\NormalTok{(V03234,V01993, V02044, V08312, V03256)}


\NormalTok{train_Depression =}\StringTok{  }\NormalTok{combo2}\OperatorTok
\StringTok{  }\KeywordTok{slice}\NormalTok{(}\DecValTok{1}\OperatorTok{:}\DecValTok{800}\NormalTok{) }\OperatorTok
\StringTok{  }\KeywordTok{select}\NormalTok{(V07876) }\OperatorTok
\StringTok{  }\NormalTok{.}\OperatorTok{$}\NormalTok{V07876}

\NormalTok{test_Depression =}\StringTok{  }\NormalTok{combo2}\OperatorTok
\StringTok{  }\KeywordTok{slice}\NormalTok{(}\DecValTok{801}\OperatorTok{:}\KeywordTok{n}\NormalTok{()) }\OperatorTok
\StringTok{  }\KeywordTok{select}\NormalTok{(V07876) }\OperatorTok
\StringTok{  }\NormalTok{.}\OperatorTok{$}\NormalTok{V07876}


\KeywordTok{set.seed}\NormalTok{(}\DecValTok{1}\NormalTok{)}
\NormalTok{knn_pred =}\StringTok{ }\KeywordTok{knn}\NormalTok{(train_obvs,}
\NormalTok{  test_obvs,}
\NormalTok{  train_Depression,}
  \DataTypeTok{k =} \DecValTok{5}\NormalTok{)}

\NormalTok{combo2}

\KeywordTok{summary}\NormalTok{(knn_pred)}
\KeywordTok{table}\NormalTok{(knn_pred, test_Depression)}
\KeywordTok{mean}\NormalTok{(knn_pred }\OperatorTok{==}\StringTok{ }\NormalTok{test_Depression)}
\end{Highlighting}
\end{Shaded}

\section{DATA IMAGES}\label{data-images}

\begin{Shaded}
\begin{Highlighting}[]
\NormalTok{sample1=}\StringTok{ }\NormalTok{data_alco}\OperatorTok
\StringTok{  }\KeywordTok{sample_n}\NormalTok{(}\DecValTok{1000}\NormalTok{)}
\NormalTok{a <-}\StringTok{ }\KeywordTok{ggplot}\NormalTok{(sample1, }\KeywordTok{aes}\NormalTok{(V08312))}\OperatorTok{+}\KeywordTok{geom_histogram}\NormalTok{()}
\NormalTok{a}
\end{Highlighting}
\end{Shaded}

\begin{verbatim}
## Don't know how to automatically pick scale for object of type labelled. Defaulting to continuous.
\end{verbatim}

\begin{verbatim}
## `stat_bin()` using `bins = 30`. Pick better value with `binwidth`.
\end{verbatim}

\begin{verbatim}
## Warning: Removed 62 rows containing non-finite values (stat_bin).
\end{verbatim}

\includegraphics{Main_Code_files/figure-latex/unnamed-chunk-21-1.pdf}

\begin{Shaded}
\begin{Highlighting}[]
\CommentTok{#Plot "a" demonstrates the data within the V08312 variable (that represents "ICD Alcohol Dependence (30 day)"), which contains categorical data, 5 is for the people respoded with a "no" and the 1 is for the people that responded with a "yes"}
\end{Highlighting}
\end{Shaded}

\begin{Shaded}
\begin{Highlighting}[]
\NormalTok{b <-}\StringTok{ }\KeywordTok{ggplot}\NormalTok{(sample1, }\KeywordTok{aes}\NormalTok{(V08515))}\OperatorTok{+}\KeywordTok{geom_histogram}\NormalTok{()}
\NormalTok{b}
\end{Highlighting}
\end{Shaded}

\begin{verbatim}
## Don't know how to automatically pick scale for object of type labelled. Defaulting to continuous.
\end{verbatim}

\begin{verbatim}
## `stat_bin()` using `bins = 30`. Pick better value with `binwidth`.
\end{verbatim}

\begin{verbatim}
## Warning: Removed 62 rows containing non-finite values (stat_bin).
\end{verbatim}

\includegraphics{Main_Code_files/figure-latex/unnamed-chunk-22-1.pdf}

\begin{Shaded}
\begin{Highlighting}[]
\CommentTok{#Plot "b" demonstrates the data within the V08515 variable (that represents "ICD Alcohol Dependence (Lifetime)"), which contains categorical data, 5 is for the people respoded with a "no" and the 1 is for the people that responded with a "yes"}
\end{Highlighting}
\end{Shaded}

\begin{Shaded}
\begin{Highlighting}[]
\NormalTok{sample2=}\StringTok{ }\NormalTok{data_su}\OperatorTok
\StringTok{  }\KeywordTok{sample_n}\NormalTok{(}\DecValTok{1000}\NormalTok{)}
\NormalTok{c <-}\StringTok{ }\KeywordTok{ggplot}\NormalTok{(sample2, }\KeywordTok{aes}\NormalTok{(V03336))}\OperatorTok{+}\KeywordTok{geom_histogram}\NormalTok{()}
\NormalTok{c}
\end{Highlighting}
\end{Shaded}

\begin{verbatim}
## Don't know how to automatically pick scale for object of type labelled. Defaulting to continuous.
\end{verbatim}

\begin{verbatim}
## `stat_bin()` using `bins = 30`. Pick better value with `binwidth`.
\end{verbatim}

\begin{verbatim}
## Warning: Removed 815 rows containing non-finite values (stat_bin).
\end{verbatim}

\includegraphics{Main_Code_files/figure-latex/unnamed-chunk-23-1.pdf}

\begin{Shaded}
\begin{Highlighting}[]
\CommentTok{#Plot "c" demonstrates the data within the V03336 variable (that represents "Used marijuana/hash before teens"), which contains categorical data, 5 is for the people respoded with a "no" and the 1 is for the people that responded with a "yes"}
\end{Highlighting}
\end{Shaded}

\begin{Shaded}
\begin{Highlighting}[]
\NormalTok{d <-}\StringTok{ }\KeywordTok{ggplot}\NormalTok{(sample2, }\KeywordTok{aes}\NormalTok{(V03290)) }\OperatorTok{+}\KeywordTok{geom_histogram}\NormalTok{()}
\NormalTok{d}
\end{Highlighting}
\end{Shaded}

\begin{verbatim}
## `stat_bin()` using `bins = 30`. Pick better value with `binwidth`.
\end{verbatim}

\begin{verbatim}
## Warning: Removed 924 rows containing non-finite values (stat_bin).
\end{verbatim}

\includegraphics{Main_Code_files/figure-latex/unnamed-chunk-24-1.pdf}

\begin{Shaded}
\begin{Highlighting}[]
\CommentTok{#Plot "d" demonstrates the data within the V03290 variable (that represents "Age 1st drinking problem occured"), which contains numerical data, the x-axis is meant to show the age range for the question.}
\end{Highlighting}
\end{Shaded}

\begin{Shaded}
\begin{Highlighting}[]
\NormalTok{sample3 =}\StringTok{ }\NormalTok{data_sph }\OperatorTok
\StringTok{  }\KeywordTok{sample_n}\NormalTok{(}\DecValTok{1000}\NormalTok{)}
\NormalTok{e <-}\StringTok{ }\KeywordTok{ggplot}\NormalTok{(sample3, }\KeywordTok{aes}\NormalTok{(V01533))}\OperatorTok{+}\KeywordTok{geom_histogram}\NormalTok{()  }\CommentTok{#social fear situation - fear vomiting}
\NormalTok{e}
\end{Highlighting}
\end{Shaded}

\begin{verbatim}
## Don't know how to automatically pick scale for object of type labelled. Defaulting to continuous.
\end{verbatim}

\begin{verbatim}
## `stat_bin()` using `bins = 30`. Pick better value with `binwidth`.
\end{verbatim}

\begin{verbatim}
## Warning: Removed 948 rows containing non-finite values (stat_bin).
\end{verbatim}

\includegraphics{Main_Code_files/figure-latex/unnamed-chunk-25-1.pdf}

\begin{Shaded}
\begin{Highlighting}[]
\NormalTok{f <-}\StringTok{ }\KeywordTok{ggplot}\NormalTok{(sample3, }\KeywordTok{aes}\NormalTok{(V01511)) }\OperatorTok{+}\StringTok{ }\KeywordTok{geom_histogram}\NormalTok{() }\CommentTok{#Shy/afraid/uncomf entering room when others are present}
\NormalTok{f}
\end{Highlighting}
\end{Shaded}

\begin{verbatim}
## Don't know how to automatically pick scale for object of type labelled. Defaulting to continuous.
\end{verbatim}

\begin{verbatim}
## `stat_bin()` using `bins = 30`. Pick better value with `binwidth`.
\end{verbatim}

\begin{verbatim}
## Warning: Removed 774 rows containing non-finite values (stat_bin).
\end{verbatim}

\includegraphics{Main_Code_files/figure-latex/unnamed-chunk-26-1.pdf}

By looking at several sample variables in our data through histogram
visualization, we have learned that in our data we have numerical
variables, but there are bunch of categorical variables, (Yes or No). In
our visualization of categorical variable, 1 indicates ``Yes'' and 5
indicates ``No''" in x-axis, and we have different ranges of scales for
y-axis. In our numerical variables, we have different scales in x and
y-axis.

\section{Numeric summaries}\label{numeric-summaries}

\subsection{Eating disorder}\label{eating-disorder}

\begin{Shaded}
\begin{Highlighting}[]
\KeywordTok{summary}\NormalTok{(data_ed[}\DecValTok{1}\OperatorTok{:}\DecValTok{25}\NormalTok{])}
\end{Highlighting}
\end{Shaded}

In this numeric summary of the first 25 variables in eating disorder the
first thing to note is the immense amount of NA's or missing data. In
another class we were discussing the issues of the different types of
missing data, missing completely at random, missing at random, and
missing not at random. I am curious if there is a way to tell what kind
of missing data we have isn't none of us are experts in the field. It
also brings up the question of what is the most appropriate way to deal
with these missing vaules? It is also likely that we will remove
Varibles with a high proportion of missing vaules\\
Another take away from the summaries is from the means. For the
variables that are only between 1 and 5 these are the yes or no
questions on the survey, 1 being yes and 5 being no, the mean can tell
us the proption of yes or nos in the varibles by seeing if the mean is
closer to 1 or 5.

\subsection{Alcohol}\label{alcohol-1}

\begin{Shaded}
\begin{Highlighting}[]
\KeywordTok{summary}\NormalTok{(data_alco[}\DecValTok{1}\OperatorTok{:}\DecValTok{5}\NormalTok{])}
\end{Highlighting}
\end{Shaded}

Three of the five varibles appear to be Yes or No survey questions with
the majority of the responses being nos, and we can tell this because
the means are extremely close to 5. This could be a concern as it means
that there is little variation in the responses and so the varibles may
not provide that much information. The other two varibles have a great
deal of missing data, and so may lack any significant information.

\subsection{Suicide}\label{suicide-1}

\begin{Shaded}
\begin{Highlighting}[]
\CommentTok{#Seriously thought about committing suicide}
\KeywordTok{summary}\NormalTok{(}\KeywordTok{subset}\NormalTok{(data_sui, }\DataTypeTok{select=}\NormalTok{V01993))}
\CommentTok{#Age 1st thought seriously about committing suicide}
\KeywordTok{summary}\NormalTok{(}\KeywordTok{subset}\NormalTok{(data_sui, }\DataTypeTok{select=}\NormalTok{V01994))}
\CommentTok{#Seriously thought about committing suicide in past 12 months}
\KeywordTok{summary}\NormalTok{(}\KeywordTok{subset}\NormalTok{(data_sui, }\DataTypeTok{select=}\NormalTok{V01995))}
\CommentTok{# Age last seriously thought about suicide}
\KeywordTok{summary}\NormalTok{(}\KeywordTok{subset}\NormalTok{(data_sui, }\DataTypeTok{select=}\NormalTok{V01996))}
\CommentTok{#Age 1st made suicide plan}
\KeywordTok{summary}\NormalTok{(}\KeywordTok{subset}\NormalTok{(data_sui, }\DataTypeTok{select=}\NormalTok{V01998))}
\end{Highlighting}
\end{Shaded}

From the numerical summaries of these variables from the Suicide subset
of the data we can see that there are some categorical variables, but
most are numerical that depend on a larger range of numbers, and in this
case the numbers represent the ages of the people involved in the
studies. With our data we learn that most people said
â\euro{}œnoâ\euro{} to seriously committing suicide both in the last 12
months and at all. But from the other variables we can see that the
people that have thought about committing suicide had these thoughts in
their 20s. There are many NA values, so the summaries donâ\euro{}™t
represent all of the observations from the data.

\subsection{Social Phobia}\label{social-phobia-1}

\begin{Shaded}
\begin{Highlighting}[]
\KeywordTok{summary}\NormalTok{(data_sph[}\DecValTok{1}\OperatorTok{:}\DecValTok{25}\NormalTok{])}
\end{Highlighting}
\end{Shaded}

In the numerical summaries in social phobia variable, there are only
four numerical variables in the first 25 variables. We can figure out
that this dataset has lots of categorical variables in here too. Also
there are lots of NAâ\euro{}™s values. The mean of each categorical
variable indicates the proportion of Yes or No in the variable. If the
mean value is close to 1, the variables are dominated by the answer Yes.
If the mean value is far from 1, the variables are dominated by NO.


\end{document}
