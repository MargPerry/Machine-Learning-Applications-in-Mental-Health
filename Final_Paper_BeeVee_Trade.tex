\documentclass[]{article}
\usepackage{lmodern}
\usepackage{amssymb,amsmath}
\usepackage{ifxetex,ifluatex}
\usepackage{fixltx2e} % provides \textsubscript
\ifnum 0\ifxetex 1\fi\ifluatex 1\fi=0 % if pdftex
  \usepackage[T1]{fontenc}
  \usepackage[utf8]{inputenc}
\else % if luatex or xelatex
  \ifxetex
    \usepackage{mathspec}
  \else
    \usepackage{fontspec}
  \fi
  \defaultfontfeatures{Ligatures=TeX,Scale=MatchLowercase}
\fi
% use upquote if available, for straight quotes in verbatim environments
\IfFileExists{upquote.sty}{\usepackage{upquote}}{}
% use microtype if available
\IfFileExists{microtype.sty}{%
\usepackage{microtype}
\UseMicrotypeSet[protrusion]{basicmath} % disable protrusion for tt fonts
}{}
\usepackage[margin=1in]{geometry}
\usepackage{hyperref}
\hypersetup{unicode=true,
            pdftitle={Predicting Depression through Machine Learning Methods},
            pdfauthor={BeeVee Trade},
            pdfborder={0 0 0},
            breaklinks=true}
\urlstyle{same}  % don't use monospace font for urls
\usepackage{color}
\usepackage{fancyvrb}
\newcommand{\VerbBar}{|}
\newcommand{\VERB}{\Verb[commandchars=\\\{\}]}
\DefineVerbatimEnvironment{Highlighting}{Verbatim}{commandchars=\\\{\}}
% Add ',fontsize=\small' for more characters per line
\usepackage{framed}
\definecolor{shadecolor}{RGB}{248,248,248}
\newenvironment{Shaded}{\begin{snugshade}}{\end{snugshade}}
\newcommand{\KeywordTok}[1]{\textcolor[rgb]{0.13,0.29,0.53}{\textbf{#1}}}
\newcommand{\DataTypeTok}[1]{\textcolor[rgb]{0.13,0.29,0.53}{#1}}
\newcommand{\DecValTok}[1]{\textcolor[rgb]{0.00,0.00,0.81}{#1}}
\newcommand{\BaseNTok}[1]{\textcolor[rgb]{0.00,0.00,0.81}{#1}}
\newcommand{\FloatTok}[1]{\textcolor[rgb]{0.00,0.00,0.81}{#1}}
\newcommand{\ConstantTok}[1]{\textcolor[rgb]{0.00,0.00,0.00}{#1}}
\newcommand{\CharTok}[1]{\textcolor[rgb]{0.31,0.60,0.02}{#1}}
\newcommand{\SpecialCharTok}[1]{\textcolor[rgb]{0.00,0.00,0.00}{#1}}
\newcommand{\StringTok}[1]{\textcolor[rgb]{0.31,0.60,0.02}{#1}}
\newcommand{\VerbatimStringTok}[1]{\textcolor[rgb]{0.31,0.60,0.02}{#1}}
\newcommand{\SpecialStringTok}[1]{\textcolor[rgb]{0.31,0.60,0.02}{#1}}
\newcommand{\ImportTok}[1]{#1}
\newcommand{\CommentTok}[1]{\textcolor[rgb]{0.56,0.35,0.01}{\textit{#1}}}
\newcommand{\DocumentationTok}[1]{\textcolor[rgb]{0.56,0.35,0.01}{\textbf{\textit{#1}}}}
\newcommand{\AnnotationTok}[1]{\textcolor[rgb]{0.56,0.35,0.01}{\textbf{\textit{#1}}}}
\newcommand{\CommentVarTok}[1]{\textcolor[rgb]{0.56,0.35,0.01}{\textbf{\textit{#1}}}}
\newcommand{\OtherTok}[1]{\textcolor[rgb]{0.56,0.35,0.01}{#1}}
\newcommand{\FunctionTok}[1]{\textcolor[rgb]{0.00,0.00,0.00}{#1}}
\newcommand{\VariableTok}[1]{\textcolor[rgb]{0.00,0.00,0.00}{#1}}
\newcommand{\ControlFlowTok}[1]{\textcolor[rgb]{0.13,0.29,0.53}{\textbf{#1}}}
\newcommand{\OperatorTok}[1]{\textcolor[rgb]{0.81,0.36,0.00}{\textbf{#1}}}
\newcommand{\BuiltInTok}[1]{#1}
\newcommand{\ExtensionTok}[1]{#1}
\newcommand{\PreprocessorTok}[1]{\textcolor[rgb]{0.56,0.35,0.01}{\textit{#1}}}
\newcommand{\AttributeTok}[1]{\textcolor[rgb]{0.77,0.63,0.00}{#1}}
\newcommand{\RegionMarkerTok}[1]{#1}
\newcommand{\InformationTok}[1]{\textcolor[rgb]{0.56,0.35,0.01}{\textbf{\textit{#1}}}}
\newcommand{\WarningTok}[1]{\textcolor[rgb]{0.56,0.35,0.01}{\textbf{\textit{#1}}}}
\newcommand{\AlertTok}[1]{\textcolor[rgb]{0.94,0.16,0.16}{#1}}
\newcommand{\ErrorTok}[1]{\textcolor[rgb]{0.64,0.00,0.00}{\textbf{#1}}}
\newcommand{\NormalTok}[1]{#1}
\usepackage{graphicx,grffile}
\makeatletter
\def\maxwidth{\ifdim\Gin@nat@width>\linewidth\linewidth\else\Gin@nat@width\fi}
\def\maxheight{\ifdim\Gin@nat@height>\textheight\textheight\else\Gin@nat@height\fi}
\makeatother
% Scale images if necessary, so that they will not overflow the page
% margins by default, and it is still possible to overwrite the defaults
% using explicit options in \includegraphics[width, height, ...]{}
\setkeys{Gin}{width=\maxwidth,height=\maxheight,keepaspectratio}
\IfFileExists{parskip.sty}{%
\usepackage{parskip}
}{% else
\setlength{\parindent}{0pt}
\setlength{\parskip}{6pt plus 2pt minus 1pt}
}
\setlength{\emergencystretch}{3em}  % prevent overfull lines
\providecommand{\tightlist}{%
  \setlength{\itemsep}{0pt}\setlength{\parskip}{0pt}}
\setcounter{secnumdepth}{0}
% Redefines (sub)paragraphs to behave more like sections
\ifx\paragraph\undefined\else
\let\oldparagraph\paragraph
\renewcommand{\paragraph}[1]{\oldparagraph{#1}\mbox{}}
\fi
\ifx\subparagraph\undefined\else
\let\oldsubparagraph\subparagraph
\renewcommand{\subparagraph}[1]{\oldsubparagraph{#1}\mbox{}}
\fi

%%% Use protect on footnotes to avoid problems with footnotes in titles
\let\rmarkdownfootnote\footnote%
\def\footnote{\protect\rmarkdownfootnote}

%%% Change title format to be more compact
\usepackage{titling}

% Create subtitle command for use in maketitle
\newcommand{\subtitle}[1]{
  \posttitle{
    \begin{center}\large#1\end{center}
    }
}

\setlength{\droptitle}{-2em}
  \title{Predicting Depression through Machine Learning Methods}
  \pretitle{\vspace{\droptitle}\centering\huge}
  \posttitle{\par}
  \author{BeeVee Trade}
  \preauthor{\centering\large\emph}
  \postauthor{\par}
  \predate{\centering\large\emph}
  \postdate{\par}
  \date{12/11/2018}


\begin{document}
\maketitle

\begin{Shaded}
\begin{Highlighting}[]
\KeywordTok{library}\NormalTok{(tidyr)}
\KeywordTok{library}\NormalTok{(dplyr)}
\KeywordTok{library}\NormalTok{(mosaic)}
\KeywordTok{library}\NormalTok{(tree)}
\end{Highlighting}
\end{Shaded}

\begin{verbatim}
## Warning: package 'tree' was built under R version 3.4.4
\end{verbatim}

\begin{Shaded}
\begin{Highlighting}[]
\KeywordTok{library}\NormalTok{(ISLR)}
\KeywordTok{library}\NormalTok{(ggplot2)}
\KeywordTok{library}\NormalTok{(class)}
\KeywordTok{library}\NormalTok{(foreign)}
\end{Highlighting}
\end{Shaded}

\begin{verbatim}
## Warning: package 'foreign' was built under R version 3.4.4
\end{verbatim}

\begin{Shaded}
\begin{Highlighting}[]
\KeywordTok{library}\NormalTok{(readr)}
\KeywordTok{library}\NormalTok{(gbm)}
\end{Highlighting}
\end{Shaded}

\begin{verbatim}
## Warning: package 'gbm' was built under R version 3.4.4
\end{verbatim}

\section{Introduction}\label{introduction}

Mental health is an elusive subject of study. It is highly
heterogeneous? and has a broad spectrum of symptoms and severity. Major
Depressive Disorders (MDD) affects about 40 million adults in the United
States, which begs the question: Can we predict whether or not someone
will develop a form of depression? We built a model that attempts to
predict if someone is depressed based on key variables. We have selected
themes that extend from social networks to substance use. The hope is
that we will be able to draw a predictive model from within these
themes. The issue of MDD has become more predanite in recent years as
the rates for diagnosis have increased. Depression affects both the
mental and physical state of an individual. It can cause feelings of
helplessness, a loss of interest in hobbies, changes in weight and
appetite, and in extreme cases can lead to death. It is a delicate
subject no matter how it is brought up and is often taboo to discuss.
Machine learning techniques will allow us to see if we can determine who
is most at risk and then be able to enact preventative measures. Within
our own community there are many who suffer from depression and there
are many more who have yet to be diagnosed. College students are at high
risk of developing depression, as there is an acute combination of high
stress, availability to substances, and a possible lack of a support
group, which all contribute to poor mental health and poor self-care.
Depression is close to many of our hearts, and addressing this issue
might help us lead better and happier lives. Therefore, we hope to
investigate the question about the determinants of major depressive
disorders (social network, environmental, personal, etc., and
comorbidities, like eating disorders) as these are relevant for both
college students and the general public. Specific questions that can be
asked: Are students that engage more time in social activity/
maintaining social network less prone to be diagnosed with MDD? Does
increased frequency of substance use increase the chance of being
diagnosed with MDD and increase the episode? As a group the key
predictors we are interested in are eating disorders, family
relationship, social network, personality traits, substance use (tobacco
and alcohol), and suicidal thoughts and tendencies.

\section{Data}\label{data}

We used a large joint dataset of three nationally representative
surveys: the National Comorbidity Survey Replication (NCS-R), the
National Survey of American Life (NSAL), and the National Latino and
Asian American Study (NLAAS). This Collaborative Psychiatric
Epidemiology Surveys (CPES) was initiated in recognition of the need for
contemporary, comprehensive epidemiological data regarding the
distributions, correlates and risk factors of mental disorders among the
general population with special emphasis on minority groups. Its data
collection covers a total of 252 geographic areas across United States
and contains a total of 20013 observations and 5543 variables. All
variables are also crosswalked under each category of mental disorder.

\section{Data Cleaning and Methods}\label{data-cleaning-and-methods}

From the numerical summaries of these variables from the Suicide subset
of the data we can see that there are some categorical variables, but
most are numerical that depend on a larger range of numbers, and in this
case the numbers represent the ages of the people involved in the
studies. With our data we learn that most people said no to seriously
committing suicide both in the last 12 months and at all. But from the
other variables we can see that the people that have thought about
committing suicide had these thoughts in their 20s. There are many NA
values, so the summaries don't represent all of the observations from
the data.

We also realize that we need to convert some numeric values to binary
categorical values, for instance, there are two numbers, 1(no) and
5(yes), are used for indicating ICD alcohol dependence (lifetime). But
we believe it will be confusing to use it as numeric and we want to
convert it into a categorical variable. This is important to make sure
about our consistency in treating variables with two categories as
categorical, and treating variables with multiple categories as numeric.
There is a huge inconsistency in the values available for observations
used in different categories. For instance, it is hard to compare the
data from variables under category eating disorder with variables from
category family cohesion due to two datasets being actually collected
from two separate surveys. Even though these two datasets were compiled
together, it shows that the observations from one survey do not
necessarily include variables from another survey. Therefore, we believe
that it is essential for us to go back to our cross-walk original
dataset again to select variables that are all from the same survey,
instead of the collaborative survey that contains observations from
other surveys.

The same problem is also shown when we tried to use tree decision as our
main model. We were very excited to use tree decision model, however,
our model turned out to be a disaster. We only got one node and aren't
sure the cause. We believe that it might be due to missing values in ou
data. Our data worked well with KNN, mainly because KNN does not have
any assumptions. The only obstacle was that we were required to filter
out observations that had any missing values, however this still left us
with over 1,500 for the method. As for logistic regression, our data met
the assumption that the response variable is binary. We also tested on
large enough data (a sample of 1670). However, we are concerned about
potential multicollinearity, such as seriously contemplating suicide and
actually attempting suicide. It may suffice that we pick only one of
these variables.

We believe that we are having major issues with the variables we are
using right now. Many numeric variables containing only two values
should be converted into categorical variables. And we need to revisit
our initial category and variable selection by making sure all variables
are from the same survey, instead of the collaborative survey. Even
though it is painful to acknowledge this mistake we made earlier on
about variable selection, we have learned a lot about reading and
understanding complex collaborative survey datasets. And then we will
try tree decision or logistic model again

However, it was at the very late stage of our data analysis did we
realise that we have imported the SPSS dataset in the wrong format due
to the wrong package. Therefore, it was no surprising that we were
frustrated with data class conversion because all of them were in the
wrong class. Thus, there was always an error here and there with
variable selection and conversion. But this is a good learning
experience in the future for us to always check the normality of the
data structure before deciding the next step.

\begin{Shaded}
\begin{Highlighting}[]
\NormalTok{newdata=}\StringTok{ }\KeywordTok{read.csv}\NormalTok{(}\StringTok{'newdata.csv'}\NormalTok{)}
\CommentTok{#filter variables with less than 25% missing values }
\NormalTok{dat.}\DecValTok{25}\NormalTok{ <-}\StringTok{ }\NormalTok{newdata[, }\OperatorTok{-}\KeywordTok{which}\NormalTok{(}\KeywordTok{colMeans}\NormalTok{(}\KeywordTok{is.na}\NormalTok{(newdata)) }\OperatorTok{>}\StringTok{ }\FloatTok{0.25}\NormalTok{)]}
\end{Highlighting}
\end{Shaded}

\begin{Shaded}
\begin{Highlighting}[]
\CommentTok{#converting the response variable into either 1 or 0 and remove observations with missing values }
\NormalTok{dat.}\DecValTok{25}\OperatorTok{$}\NormalTok{X <-}\StringTok{ }\OtherTok{NULL}
\NormalTok{dat.}\DecValTok{25}\OperatorTok{$}\NormalTok{V07657 <-}\StringTok{ }\OtherTok{NULL}
\NormalTok{dat.}\DecValTok{25}\OperatorTok{$}\NormalTok{V07876 <-}\StringTok{ }\OtherTok{NULL}

\NormalTok{dat.}\DecValTok{25}\OperatorTok{$}\NormalTok{V07655 <-}\StringTok{ }\KeywordTok{as.numeric}\NormalTok{(dat.}\DecValTok{25}\OperatorTok{$}\NormalTok{V07655)}
\NormalTok{dat.}\DecValTok{25}\OperatorTok{$}\NormalTok{V07655[dat.}\DecValTok{25}\OperatorTok{$}\NormalTok{V07655 }\OperatorTok{==}\StringTok{ "2"}\NormalTok{] <-}\StringTok{ }\DecValTok{0} 

\NormalTok{dat.}\DecValTok{25}\NormalTok{ <-}\StringTok{ }\NormalTok{dat.}\DecValTok{25} \OperatorTok
\StringTok{  }\KeywordTok{filter}\NormalTok{(}\KeywordTok{complete.cases}\NormalTok{(.))}
\end{Highlighting}
\end{Shaded}

\section{Results}\label{results}

When we first ran our decision tree model we found that we had created
trees with no prediction values as all the nodes are zero. This is what
lead to the second wave of data cleaning. After we cleaned our data for
the second time, we found through the decision tree model that there
were three main predictors of endorsed depressive 12 month episode were
number of years school the mother completed, religious preference, and
highest grade of school or college completed. We found that the
individual tree models predicted with 59\% overall accuracy rate, a 49
\% accuracy rate for the model to predict if someone will be depressed,
and 62 \% accuracy rate for the model to predict if someone will not be
depressed. However when we ran the boosting trees our model predicts
with about 60\% accuracy when using test data.

\begin{Shaded}
\begin{Highlighting}[]
\CommentTok{#because the orignial percentage of depressed cases in the overall dataset is very small (only 8%).}
\CommentTok{#we restore the balance by adding one more condition: whether the person has any serious suicidal thoughts. Filter out the observations in which the answer is YES. }

\CommentTok{#the new percentage is 20%}

\NormalTok{dat.}\DecValTok{25}\NormalTok{ <-}\StringTok{ }\NormalTok{dat.}\DecValTok{25} \OperatorTok
\StringTok{  }\KeywordTok{filter}\NormalTok{(dat.}\DecValTok{25}\OperatorTok{$}\NormalTok{V01993 }\OperatorTok{==}\StringTok{ "YES"}\NormalTok{)}
\NormalTok{dat.}\DecValTok{25}\OperatorTok{$}\NormalTok{V01993 <-}\StringTok{ }\OtherTok{NULL}

\CommentTok{#there are 319 out of 1597(20%) are diagnosed with depression }
\end{Highlighting}
\end{Shaded}

\section{Using classification tree}\label{using-classification-tree}

\begin{Shaded}
\begin{Highlighting}[]
\CommentTok{#turning logi and numeric into factor }

\NormalTok{dat.}\DecValTok{25}\OperatorTok{$}\NormalTok{V03221 <-}\StringTok{ }\KeywordTok{as.factor}\NormalTok{(dat.}\DecValTok{25}\OperatorTok{$}\NormalTok{V03221 )}
\NormalTok{dat.}\DecValTok{25}\OperatorTok{$}\NormalTok{V03223 <-}\StringTok{ }\KeywordTok{as.factor}\NormalTok{(dat.}\DecValTok{25}\OperatorTok{$}\NormalTok{V03223 )}
\NormalTok{dat.}\DecValTok{25}\OperatorTok{$}\NormalTok{V03224 <-}\StringTok{ }\KeywordTok{as.factor}\NormalTok{(dat.}\DecValTok{25}\OperatorTok{$}\NormalTok{V03224 )}
\NormalTok{dat.}\DecValTok{25}\OperatorTok{$}\NormalTok{V03225 <-}\StringTok{ }\KeywordTok{as.factor}\NormalTok{(dat.}\DecValTok{25}\OperatorTok{$}\NormalTok{V03225 )}
\NormalTok{dat.}\DecValTok{25}\OperatorTok{$}\NormalTok{V03226 <-}\StringTok{ }\KeywordTok{as.factor}\NormalTok{(dat.}\DecValTok{25}\OperatorTok{$}\NormalTok{V03226 )}
\NormalTok{dat.}\DecValTok{25}\OperatorTok{$}\NormalTok{V03227 <-}\StringTok{ }\KeywordTok{as.factor}\NormalTok{(dat.}\DecValTok{25}\OperatorTok{$}\NormalTok{V03227 )}
\NormalTok{dat.}\DecValTok{25}\OperatorTok{$}\NormalTok{V03228 <-}\StringTok{ }\KeywordTok{as.factor}\NormalTok{(dat.}\DecValTok{25}\OperatorTok{$}\NormalTok{V03228 )}
\NormalTok{dat.}\DecValTok{25}\OperatorTok{$}\NormalTok{V03229 <-}\StringTok{ }\KeywordTok{as.factor}\NormalTok{(dat.}\DecValTok{25}\OperatorTok{$}\NormalTok{V03229 )}
\NormalTok{dat.}\DecValTok{25}\OperatorTok{$}\NormalTok{V03230 <-}\StringTok{ }\KeywordTok{as.factor}\NormalTok{(dat.}\DecValTok{25}\OperatorTok{$}\NormalTok{V03230 )}
\NormalTok{dat.}\DecValTok{25}\OperatorTok{$}\NormalTok{V03231 <-}\StringTok{ }\KeywordTok{as.factor}\NormalTok{(dat.}\DecValTok{25}\OperatorTok{$}\NormalTok{V03231 )}
\NormalTok{dat.}\DecValTok{25}\OperatorTok{$}\NormalTok{V09389 <-}\StringTok{ }\KeywordTok{as.factor}\NormalTok{(dat.}\DecValTok{25}\OperatorTok{$}\NormalTok{V09389)}
\NormalTok{dat.}\DecValTok{25}\OperatorTok{$}\NormalTok{V07655 <-}\StringTok{ }\KeywordTok{as.factor}\NormalTok{(dat.}\DecValTok{25}\OperatorTok{$}\NormalTok{V07655 )}

\KeywordTok{set.seed}\NormalTok{(}\DecValTok{1}\NormalTok{)}
\NormalTok{train =}\StringTok{ }\NormalTok{dat.}\DecValTok{25}\OperatorTok
\StringTok{  }\KeywordTok{sample_frac}\NormalTok{(.}\DecValTok{5}\NormalTok{)}
\NormalTok{test =}\StringTok{ }\NormalTok{dat.}\DecValTok{25}\OperatorTok
\StringTok{  }\KeywordTok{setdiff}\NormalTok{(train)}

\NormalTok{tree1 <-}\StringTok{ }\KeywordTok{tree}\NormalTok{(V07655 }\OperatorTok{~}\NormalTok{., }\DataTypeTok{data=}\NormalTok{train)}
\KeywordTok{summary}\NormalTok{(tree1)}
\end{Highlighting}
\end{Shaded}

\begin{verbatim}
## 
## Classification tree:
## tree(formula = V07655 ~ ., data = train)
## Variables actually used in tree construction:
## [1] "V03085" "V09389" "V09414" "V09409"
## Number of terminal nodes:  5 
## Residual mean deviance:  0.9448 = 749.3 / 793 
## Misclassification error rate: 0.1942 = 155 / 798
\end{verbatim}

\begin{Shaded}
\begin{Highlighting}[]
\KeywordTok{plot}\NormalTok{(tree1)}
\KeywordTok{text}\NormalTok{(tree1, }\DataTypeTok{pretty =} \DecValTok{0}\NormalTok{)}
\end{Highlighting}
\end{Shaded}

\includegraphics{Final_Paper_BeeVee_Trade_files/figure-latex/unnamed-chunk-5-1.pdf}

\begin{Shaded}
\begin{Highlighting}[]
\NormalTok{tree_pred =}\StringTok{ }\KeywordTok{predict}\NormalTok{(tree1, test, }\DataTypeTok{type =} \StringTok{"class"}\NormalTok{)}
\KeywordTok{table}\NormalTok{(tree_pred, test}\OperatorTok{$}\NormalTok{V07655)}
\end{Highlighting}
\end{Shaded}

\begin{verbatim}
##          
## tree_pred   0   1
##         0 629 163
##         1   0   0
\end{verbatim}

So it seems like our tree model would not classify any observations into
``endorsed''. The accuration rate is \(629/(629+163) = 79.5%\). Maybe it
finds that it is the easiest and safest to not classify anyone as
depressed. However, this model yields no use to real-life application if
it can not predict anyone to be endorsed with depression.

\section{Using boosting tree}\label{using-boosting-tree}

\begin{Shaded}
\begin{Highlighting}[]
\NormalTok{dat.}\DecValTok{25}\OperatorTok{$}\NormalTok{V07655 <-}\StringTok{ }\KeywordTok{as.numeric}\NormalTok{(dat.}\DecValTok{25}\OperatorTok{$}\NormalTok{V07655)}
\CommentTok{#here because it's tricky to transfer a factor to numeric and change to 1 again}
\CommentTok{#we use 0 to denote endorsed depression }
\CommentTok{#we use 1 to denote not endorsed with depression }
\NormalTok{dat.}\DecValTok{25}\OperatorTok{$}\NormalTok{V07655[dat.}\DecValTok{25}\OperatorTok{$}\NormalTok{V07655 }\OperatorTok{==}\StringTok{ "2"}\NormalTok{] <-}\StringTok{ }\DecValTok{0}

\KeywordTok{set.seed}\NormalTok{(}\DecValTok{2}\NormalTok{)}
\NormalTok{train =}\StringTok{ }\NormalTok{dat.}\DecValTok{25}\OperatorTok
\StringTok{  }\KeywordTok{sample_frac}\NormalTok{(.}\DecValTok{5}\NormalTok{)}
\NormalTok{test =}\StringTok{ }\NormalTok{dat.}\DecValTok{25}\OperatorTok
\StringTok{  }\KeywordTok{setdiff}\NormalTok{(train)}

\NormalTok{boost.tree =}\StringTok{ }\KeywordTok{gbm}\NormalTok{(V07655}\OperatorTok{~}\NormalTok{., }\DataTypeTok{data =}\NormalTok{ train, }\DataTypeTok{distribution =} \StringTok{"bernoulli"}\NormalTok{, }\DataTypeTok{n.trees =} \DecValTok{5000}\NormalTok{, }\DataTypeTok{shrinkage =} \FloatTok{0.01}\NormalTok{)}
\KeywordTok{summary}\NormalTok{(boost.tree)}
\end{Highlighting}
\end{Shaded}

\includegraphics{Final_Paper_BeeVee_Trade_files/figure-latex/unnamed-chunk-7-1.pdf}

\begin{verbatim}
##           var     rel.inf
## V09414 V09414 28.88226936
## V09389 V09389 28.43383360
## V03085 V03085 27.69975993
## V09409 V09409  3.42879006
## V08311 V08311  2.78094329
## V09413 V09413  2.39421819
## V03223 V03223  1.33302562
## V03230 V03230  0.86225369
## V03333 V03333  0.81172284
## V03224 V03224  0.71726434
## V03221 V03221  0.48757214
## V03226 V03226  0.39903275
## V03225 V03225  0.32983124
## V03228 V03228  0.26435754
## V03331 V03331  0.25236403
## V03330 V03330  0.22303949
## V03332 V03332  0.21786383
## V03227 V03227  0.18739232
## V08515 V08515  0.10480360
## V03231 V03231  0.08973823
## V05700 V05700  0.06207487
## V03229 V03229  0.03784905
## V08312 V08312  0.00000000
\end{verbatim}

Here we can see that V09389 (number of years mother has completed),
V09414(religious preference) and V03085 (number of education years) are
three very important variables. We can also produce partial dependence
plots for these two variables. These plots illustrate the marginal
effect of the selected variables on the response after integrating out
the other variables.

\begin{Shaded}
\begin{Highlighting}[]
\KeywordTok{par}\NormalTok{(}\DataTypeTok{mfrow =} \KeywordTok{c}\NormalTok{(}\DecValTok{1}\NormalTok{,}\DecValTok{2}\NormalTok{))}
\KeywordTok{plot}\NormalTok{(boost.tree, }\DataTypeTok{i =} \StringTok{"V09389"}\NormalTok{)}
\end{Highlighting}
\end{Shaded}

\includegraphics{Final_Paper_BeeVee_Trade_files/figure-latex/unnamed-chunk-8-1.pdf}

\begin{Shaded}
\begin{Highlighting}[]
\KeywordTok{plot}\NormalTok{(boost.tree, }\DataTypeTok{i =} \StringTok{"V09414"}\NormalTok{)}
\end{Highlighting}
\end{Shaded}

\includegraphics{Final_Paper_BeeVee_Trade_files/figure-latex/unnamed-chunk-8-2.pdf}

\begin{Shaded}
\begin{Highlighting}[]
\KeywordTok{plot}\NormalTok{(boost.tree, }\DataTypeTok{i =} \StringTok{"V03085"}\NormalTok{)}
\end{Highlighting}
\end{Shaded}

\includegraphics{Final_Paper_BeeVee_Trade_files/figure-latex/unnamed-chunk-8-3.pdf}

Use the boosting model to predict the response on the test data.

\begin{Shaded}
\begin{Highlighting}[]
\NormalTok{boost.pred =}\StringTok{ }\KeywordTok{predict}\NormalTok{(boost.tree, test, }\DataTypeTok{type =} \StringTok{"response"}\NormalTok{, }\DataTypeTok{n.trees=}\DecValTok{5000}\NormalTok{)}
\KeywordTok{histogram}\NormalTok{(boost.pred)}
\end{Highlighting}
\end{Shaded}

\includegraphics{Final_Paper_BeeVee_Trade_files/figure-latex/unnamed-chunk-9-1.pdf}

\begin{Shaded}
\begin{Highlighting}[]
\CommentTok{#Remember we use 0 to denote endorsed depression }
\CommentTok{#we use 1 to denote not endorsed with depression }
\end{Highlighting}
\end{Shaded}

So the result shows that the majority have more than 80\% of chance to
be not endorsed with depression.

\begin{Shaded}
\begin{Highlighting}[]
\CommentTok{#we use 0.8 as cutoff point and we switch the denotation again.}
\CommentTok{#1 means endorsed with depression}
\CommentTok{#0 means not endorsed with depression }
\NormalTok{boost.pred.class =}\StringTok{ }\KeywordTok{ifelse}\NormalTok{(boost.pred }\OperatorTok{>}\StringTok{ }\FloatTok{0.85}\NormalTok{, }\DecValTok{0}\NormalTok{,}\DecValTok{1}\NormalTok{)}
\KeywordTok{table}\NormalTok{(boost.pred.class, test}\OperatorTok{$}\NormalTok{V07655, }\DataTypeTok{dnn =} \KeywordTok{c}\NormalTok{(}\StringTok{"pred"}\NormalTok{, }\StringTok{"test"}\NormalTok{))}
\end{Highlighting}
\end{Shaded}

\begin{verbatim}
##     test
## pred   0   1
##    0  62 245
##    1  90 396
\end{verbatim}

\((396+62)/(396+62+245+90) = 0.5775536\) so we have around 57\%
accuracy.

\section{Discussion}\label{discussion}

It is interesting that for unknown reasons, our classification tree with
20\% observations endorsed with depression didn't assign any test data
into depressed. We wonder how it would be different if we raise the
percentage to 50\% by selecting an equal number of unendorsed cases with
endorsed cases.

\begin{Shaded}
\begin{Highlighting}[]
\NormalTok{notendorsed <-}\StringTok{ }\NormalTok{dat.}\DecValTok{25} \OperatorTok
\StringTok{  }\KeywordTok{filter}\NormalTok{(dat.}\DecValTok{25}\OperatorTok{$}\NormalTok{V07655 }\OperatorTok{==}\StringTok{ "1"}\NormalTok{)}

\NormalTok{notendorsed =}\StringTok{ }\NormalTok{notendorsed[}\KeywordTok{sample}\NormalTok{(}\KeywordTok{nrow}\NormalTok{(notendorsed), }\DecValTok{319}\NormalTok{), ]}
\NormalTok{endorsed.dep <-}\StringTok{ }\NormalTok{dat.}\DecValTok{25} \OperatorTok
\StringTok{  }\KeywordTok{filter}\NormalTok{(dat.}\DecValTok{25}\OperatorTok{$}\NormalTok{V07655 }\OperatorTok{==}\StringTok{ "0"}\NormalTok{)}

\NormalTok{newtrain =}\StringTok{ }\KeywordTok{rbind}\NormalTok{(endorsed.dep, notendorsed)}
\NormalTok{newtrain}\OperatorTok{$}\NormalTok{V07655 =}\StringTok{ }\KeywordTok{as.factor}\NormalTok{(newtrain}\OperatorTok{$}\NormalTok{V07655 )}

\KeywordTok{set.seed}\NormalTok{(}\DecValTok{6}\NormalTok{)}
\NormalTok{train =}\StringTok{ }\NormalTok{newtrain}\OperatorTok
\StringTok{  }\KeywordTok{sample_frac}\NormalTok{(.}\DecValTok{5}\NormalTok{)}
\NormalTok{test =}\StringTok{ }\NormalTok{newtrain}\OperatorTok
\StringTok{  }\KeywordTok{setdiff}\NormalTok{(train)}

\NormalTok{tree4 <-}\StringTok{ }\KeywordTok{tree}\NormalTok{(V07655 }\OperatorTok{~}\NormalTok{., }\DataTypeTok{data=}\NormalTok{train)}
\KeywordTok{summary}\NormalTok{(tree4)}
\end{Highlighting}
\end{Shaded}

\begin{verbatim}
## 
## Classification tree:
## tree(formula = V07655 ~ ., data = train)
## Variables actually used in tree construction:
## [1] "V09414" "V03085" "V09389" "V09409" "V08515" "V03331" "V03228"
## Number of terminal nodes:  22 
## Residual mean deviance:  0.9432 = 280.1 / 297 
## Misclassification error rate: 0.2633 = 84 / 319
\end{verbatim}

\begin{Shaded}
\begin{Highlighting}[]
\KeywordTok{plot}\NormalTok{(tree4)}
\KeywordTok{text}\NormalTok{(tree4, }\DataTypeTok{pretty =} \DecValTok{0}\NormalTok{)}
\end{Highlighting}
\end{Shaded}

\includegraphics{Final_Paper_BeeVee_Trade_files/figure-latex/unnamed-chunk-11-1.pdf}

\begin{Shaded}
\begin{Highlighting}[]
\NormalTok{tree_pred =}\StringTok{ }\KeywordTok{predict}\NormalTok{(tree4, test, }\DataTypeTok{type =} \StringTok{"class"}\NormalTok{)}
\KeywordTok{table}\NormalTok{(tree_pred, test}\OperatorTok{$}\NormalTok{V07655)}
\end{Highlighting}
\end{Shaded}

\begin{verbatim}
##          
## tree_pred  0  1
##         0 91 90
##         1 62 75
\end{verbatim}

\begin{Shaded}
\begin{Highlighting}[]
\CommentTok{#correct classfication rate = (69+83)/319 = 0.476489}
\end{Highlighting}
\end{Shaded}

As we can see above, afte we change the balance of depressed and not
depressed observations in the train and test dataset, we can easily use
classfication tree for seemingly good results. However, this is not a
random sample. We are curious to see how this will work for our original
dataset, and we ran this model again. Surprisingly, the tree actually
correctly classfy 2 of them as depressed.

\begin{Shaded}
\begin{Highlighting}[]
\KeywordTok{set.seed}\NormalTok{(}\DecValTok{98}\NormalTok{)}
\NormalTok{dat.}\DecValTok{25}\OperatorTok{$}\NormalTok{V07655 =}\StringTok{ }\KeywordTok{as.factor}\NormalTok{(dat.}\DecValTok{25}\OperatorTok{$}\NormalTok{V07655 )}
\NormalTok{train =}\StringTok{ }\NormalTok{dat.}\DecValTok{25}\OperatorTok
\StringTok{  }\KeywordTok{sample_frac}\NormalTok{(.}\DecValTok{5}\NormalTok{)}
\NormalTok{test =}\StringTok{ }\NormalTok{dat.}\DecValTok{25}\OperatorTok
\StringTok{  }\KeywordTok{setdiff}\NormalTok{(train)}

\NormalTok{tree4 <-}\StringTok{ }\KeywordTok{tree}\NormalTok{(V07655 }\OperatorTok{~}\NormalTok{., }\DataTypeTok{data=}\NormalTok{train)}
\KeywordTok{plot}\NormalTok{(tree4)}
\KeywordTok{text}\NormalTok{(tree4, }\DataTypeTok{pretty =} \DecValTok{0}\NormalTok{)}
\end{Highlighting}
\end{Shaded}

\includegraphics{Final_Paper_BeeVee_Trade_files/figure-latex/unnamed-chunk-12-1.pdf}

\begin{Shaded}
\begin{Highlighting}[]
\NormalTok{tree_pred =}\StringTok{ }\KeywordTok{predict}\NormalTok{(tree4, test, }\DataTypeTok{type =} \StringTok{"class"}\NormalTok{)}
\KeywordTok{table}\NormalTok{(tree_pred, test}\OperatorTok{$}\NormalTok{V07655)}
\end{Highlighting}
\end{Shaded}

\begin{verbatim}
##          
## tree_pred   0   1
##         0   2  18
##         1 155 619
\end{verbatim}

\begin{Shaded}
\begin{Highlighting}[]
\CommentTok{#correct prediction rate (2+619)/794 = 78% }
\CommentTok{#note: 1 is not depressed and 0 is depressed }
\end{Highlighting}
\end{Shaded}

In all, our biggest challenge when building the model was trying to find
a balance so that tree would not overpredict or underpredict. At first
we had a tree with 80\% accuracy, however this high level of accuracy
was because all of the observations were predicted to have no
depression, causing most of the observations to be ``properly
predicted.'' To try to have the tree have a somewhat small amount of
prediction, we filtered for specific data as a form of control. We had a
sample that only included observations which experienced suicidal
thoughts, and then included artificially selected observations that were
positive for depression. However, the model for the first time still
gives no prediction of depression and for the second time, gives 2
predictions of depression. After we artificially increase the percentage
of endorsed depressive individuals among all observations from 20\% to
50\%, we can see a better result as shown above. However, it shouldn't
take the researcher to purposely change the ratio to make the model
work, instead, we should reflect if there is any problem with our model,
our observation filtering, and our variable selection. It is likely that
our model just has really weak predictors with little to none predicting
power of depression. Therefore, for every predictor, the tree classify a
very small group of people into depressed, and therefore, even less
people will be predicted to be depressed using the second predictor.
Thus, at the end, no observation will be left to be classified as
depressed. Nonetheless, we can still think about another way to increase
the percentage by identifying another strong predictor, something
similar to ``suicidal thoughts'' to increase the balance in a more
natural way. Luckily, we can still identify three relatively stronger
predictors for depression. For future research, we may consider ways in
which we can test a range of ratio of the two levels of response
variable to see where the key cutoff lies.

Last, we hope to revisit our original giant dataset to examine the data
structure more closely. It is likely that we have omitted lots of
important information by solely filtering all the variables and
observations based on whether it has any missing values or not. We
should look at more data cleaning strategies and to know specifically
about how to deal with missing values from different types variables.
There are also some numeric variables in the original dataset that we
may consider using in the future. It is also essential to revisit our
research question to see what is the question that we are really
interested in investigating. Randomly looking for variables and seeking
any correlation is like looking for a needle in the sea. It's not
particularly fruitful without having a more specific goal in mind. This
is especially critical for how we are going to choose variables of
interest, and then how we should weigh different methods of data
cleaning.


\end{document}
